\documentclass[11pt]{article}
\usepackage[margin=1in]{geometry}
\usepackage{graphicx}
\usepackage{booktabs}
\usepackage{hyperref}
\usepackage{amsmath}

\title{NYC Taxi Arrival Modeling \\ \large Poisson vs Negative-Binomial Diagnostics and Tail Bounds}
\author{UCSD ECE225A Project Notes}
\date{\today}

\begin{document}
\maketitle

\section{Dataset Overview}
We analyze NYC TLC Yellow Taxi trips (January 2024 parquet) focusing on Manhattan pickups. Trips are bucketed into hourly windows and labeled by weekday/weekend and rush/off-peak (rush defined via CLI flags, e.g., 7--8 and 17--18). Zone metadata comes from \texttt{taxi\_zone\_lookup.csv} plus centroid calculations described in \texttt{docs/data\_prep.md}.

\section{Poisson Diagnostics}
We first tested a homogeneous Poisson model (rate $\lambda$ per zone/cohort bucket). Dispersion indexes (variance/mean) ranged from 3 to 60, rejecting the Poisson hypothesis (chi-square p-values $\ll 0.05$). Figure~\ref{fig:poisson-mismatch} shows Midtown Center weekday rush with a spike at low counts unmodeled by Poisson.

\begin{figure}[h]
    \centering
    \includegraphics[width=0.85\textwidth]{figures/poisson_midtown_rush.pdf}
    \caption{Midtown Center weekday rush: Poisson expectation fails to match heavy tail / multi-modal distribution (exported from \texttt{scripts/analyze\_manhattan\_poisson.py}).}
    \label{fig:poisson-mismatch}
\end{figure}

\section{Negative-Binomial Fit}
Given over-dispersion, we estimate NB parameters via moments:
\[
    r = \frac{\mu^2}{\sigma^2 - \mu}, \qquad p = \frac{r}{r + \mu}
\]
Histograms overlay Poisson (orange) vs NB (red). NB curves align with both the low-count spike and tail. Table~\ref{tab:nb-stats} excerpt demonstrates the diagnostics (full JSON at \texttt{outputs/manhattan\_poisson/manhattan\_poisson.json}).

\begin{table}[h]
    \centering
    \begin{tabular}{lcccc}
    \toprule
    Zone & Cohort & Dispersion & NB $r$ & NB chi-square $p$ \\
    \midrule
    Two Bridges / Seward Park & weekend rush & 3.68 & 1.39 & 0.987 \\
    Roosevelt Island & weekday offpeak & 1.36 & 0.64 & 0.951 \\
    Hudson Sq & weekend offpeak & 8.20 & 3.10 & 0.939 \\
    Washington Heights North & weekend offpeak & 1.38 & 1.49 & 0.925 \\
    \bottomrule
    \end{tabular}
    \caption{Selected NB moment estimates (source: \texttt{outputs/report\_manhattan/manhattan\_poisson.json}).}
    \label{tab:nb-stats}
\end{table}

\section{Tail Bounds}
Using \texttt{scripts/analyze\_tail\_bounds.py} we compute exceedance probability for threshold $k = \alpha \mu$ (default $\alpha=1.5$) and compare:
\begin{itemize}
    \item Empirical tail probability from bucket counts.
    \item Exact Poisson and NB survival functions.
    \item Markov, Cantelli (one-sided Chebyshev), Chernoff (Poisson + NB MGF search), and Hoeffding-style bounds.
\end{itemize}

Figure~\ref{fig:tail} highlights Manhattan zones with the highest empirical risk; NB Chernoff tracks the true tail much closer than classical Poisson-based bounds.

\begin{table}[h]
    \centering
    \begin{tabular}{lcccc}
    \toprule
    Zone (weekend rush) & $\mu$ & Threshold ($1.5\mu$) & Empirical tail & NB tail \\
    \midrule
    Penn Station / Madison Sq West & 126.9 & 190.3 & 0.44 & 0.21 \\
    Upper West Side South & 135.4 & 203.2 & 0.44 & 0.21 \\
    Greenwich Village North & 51.8 & 77.6 & 0.44 & 0.23 \\
    Union Sq & 85.6 & 128.4 & 0.44 & 0.22 \\
    \bottomrule
    \end{tabular}
    \caption{Tail-risk summary (source: \texttt{outputs/tail\_bounds/tail\_bounds.csv}). NB tail tracks empirical exceedance, while Poisson tail $\approx 10^{-7}$ and Markov/Cantelli bounds stay near 0.67.}
    \label{tab:tail}
\end{table}

\begin{figure}[h]
    \centering
    \includegraphics[width=0.9\textwidth]{figures/tail_bounds_weekday_rush.pdf}
    \caption{Tail probability vs inequality bounds (weekday rush, top zones).}
    \label{fig:tail}
\end{figure}

The CSV output (\texttt{outputs/tail\_bounds/tail\_bounds.csv}) includes columns \texttt{empirical\_tail}, \texttt{poisson\_tail}, \texttt{nb\_tail}, and bounds for each zone, enabling probability statements like:
\[
    P\big(N \ge 1.5\mu \big) \approx 0.03 \text{ (empirical)}; \quad
    \text{Poisson Chernoff} \le 0.11; \quad
    \text{NB Chernoff} \le 0.05.
\]

\section{Compiling the Report}
From the repo root:
\begin{verbatim}
cd docs/report
latexmk -pdf main.tex
\end{verbatim}
Ensure \texttt{latexmk} (TeX Live/MacTeX) is installed; it handles bibliography-less builds automatically. Figures referenced should be exported (e.g., save Plotly HTML as PNG via the GUI or \texttt{kaleido}) into \texttt{docs/report/figures/}.

\section{Next Steps}
\begin{enumerate}
    \item Fit hierarchical NB (Empirical Bayes Gamma prior on $r$) to stabilize low-volume zones.
    \item Extend tail analysis to percentile thresholds and integrate results into the course dashboard.
    \item Document Chernoff/Hoeffding derivations explicitly for the report appendix.
\end{enumerate}

\end{document}
